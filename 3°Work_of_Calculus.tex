\documentclass[a4paper]{exam}
\usepackage[portuguese]{babel}
\usepackage[utf8]{inputenc}
\printanswers

\usepackage{url}
\usepackage{endnotes}

\def\enotesize{\normalsize}
\def\makeenmark{\relax}
\def\notesname{Answers}
\def\answer#1{\endnotetext{\vspace*{-3.5ex}\begin{solution}#1\end{solution}\unskip}}
\def\theanswers{\theendnotes \medskip}
\newcounter{solution}
\stepcounter{solution}

\renewcommand{\solutiontitle}{\noindent\textbf{Solution \arabic{solution}:}\enspace\stepcounter{solution}}


\usepackage[letterpaper,top=2cm,bottom=2cm,left=3cm,right=3cm,marginparwidth=1.75cm]{geometry}
\usepackage{amsmath}
\usepackage{amssymb}
\usepackage{unicode-math}
\usepackage{graphicx}
\usepackage[colorlinks=true, allcolors=blue]{hyperref}
\begin{document}
\begin{figure}
\centering
\includegraphics[width=0.3\textwidth]{ifce.png}
\end{figure}

\title{Instituto de Educação de Ciências e Tecnologia do Ceará}
\author{José Francimi Vasconcelos de Queiroz Júnior}
\date{\today}



\maketitle

\section{Trabalho de Cálculo}
\subsection{Questões}
\begin{questions}

    

\question Faça o que se pede em cada item abaixo: 

\begin{parts}

\part {Sejam f e g funções deriváveis em a. Mostre que (fg)'(a)=f'(a) g(a)+f(a) g'(a);}
\answer{(a) \[(fg)(a)= f'(a)g(a)+f(a)g'(a) \Rightarrow \lim_{x \to a}\frac{f(x)g(x)-f(x)g(a)}{x-a} \Rightarrow \]
\[(fg)'(a)=\frac{f(x)g(x)-f(a)g(a)+f(a)g(x)-f(a)g(x)}{x-a}=\] 
 \[=\lim_{x \to a}\frac{g(x)(f(x)-f(a))}{x-a}+\frac{f(a)(g(x)-g(a))}{x-a}=\]
 \[=\lim_{x \to a}f'(x)g(x)+f(a)g(x)'=f'(a)g(a)+f(a)g'(a) \]
\newline (b)\[\text{Se } g(x)h(x) = w(x) \Rightarrow [f(x)w(x)]'=f'(x)w(x)+f(x)w'(x)\]
\[\text{Se }w(x)=g(x)h(x) \Rightarrow [f(x)g(x)h(x)]'=f'(x)g(x)h(x)+f(x)(g'(x)h(x)+g(x)h'(x))\Rightarrow\]
\[[f(x)g(x)h(x)]'=f'(x)g(x)h(x)+f(x)g'(x)h(x)+f(x)g(x)h'(x)\]} 

\part {Sejam f,g e h três funções deriváveis. Mostre que:}
\[[f(x)g(x)h(x)]'=f'(x)g(x)h(x)+f(x)g'(x)h(x)+f(x)g(x)h'(x)\]
%Resposta da b começa na linha 60
\end{parts}

\question Sejam f e g funções deriváveis em a, com \(g(a) \neq 0\). Mostre que \(\left(\frac{f}{g}\right)'(a)=\frac{f'(a)g(a)-f(a)g'(a)}{[g(a)]^2}\)

\answer{\[h(x)=\frac{f(x)}{g(x)} \Rightarrow h(x)g(x)=f(x)\Rightarrow\]
\[h'(a)g(a)+h(a)g'(a)=f'(a) \Rightarrow \frac{h'(a)f'(a) - h(a)g'(a)}{g(a)}\]
\[\text{Se }h(a)=\frac{f(x)}{g(x)}\Rightarrow \text{então } h(a)=\frac{\frac{f'(a)}{1}-\frac{f(a)}{g(a)}\frac{g'(a)}{1}}{g(a)}=\]
\[=\frac{f'(a)g(a)-f(a)g'(a)}{[g(a)]^2}\]}

\question Seja \(f(x) > 0.\) Mostre que \([f(x)^{g(x)}]'=f(x)^{g(x)}g'(x)ln(f(x))+g(x)f(x)^{g(x)-1}f'(x)\)

\answer{\[y=f(x)^{g(x)}\Rightarrow ln(y)=ln(f(x)^{g(x)})\Rightarrow ln(y)'=[g(x)ln(f(x))]'\Rightarrow \]
\[\frac{1}{y}y'=g'(x)ln(f(x))+g(x)ln(f(x))'\]
\[y'=f(x)^{g(x)}g'(x)ln(f(x))+f(x)^{g(x)}g(x)\frac{1}{f(x)}f'(x)\Rightarrow\]
\[[f(x)^{g(x)}]'=f(x)^{g(x)}g'(x)ln(f(x))+f(x)^{g(x)-1}g(x)f'(x)\]}

\question Seja g: \(\mathbb{R} \to \mathbb{R}\) uma  função diferenciável, tal que, \(g(1)=2 \text{ e } g'(1)=3.\) Calcule f'(0), sabendo que \(f(x)=e^xg(3x+1)\)

\answer{\[\text{Derivando o f(x):}\]
\[f(x)=e^x(3x+1) \Rightarrow f'(x)=[e^x]'g(3x+1)+e^x[g(3x+1)]' \Rightarrow \]
\[f'(x)=e^xg(3x+1)+e^xg'(3x+1)3 \Rightarrow f'(x)=e^x[g(3x+1)+3g'(3x+1)]\]
\[\text{Então para f'(0):}\]
\[f'(0)=e^0[g(3 \cdot 0+1)+3g'(3 \cdot 0+1)] \Rightarrow f'(0)=1[(0+1)+3g'(0+1)]\Rightarrow\]
\[f'(0)=g(1)+3g'(1)=2+3\cdot3=11 \]}

\end{questions}

\theanswers

\clearpage


\end{document}