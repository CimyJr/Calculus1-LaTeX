\documentclass[a4paper]{exam}
\usepackage[portuguese]{babel}
\usepackage[utf8]{inputenc}
\printanswers

\usepackage{url}
\usepackage{endnotes}

\def\enotesize{\normalsize}
\def\makeenmark{\relax}
\def\notesname{Answers}
\def\answer#1{\endnotetext{\vspace*{-3.5ex}\begin{solution}#1\end{solution}\unskip}}
\def\theanswers{\theendnotes \medskip}
\newcounter{solution}
\stepcounter{solution}

\renewcommand{\solutiontitle}{\noindent\textbf{Solution \arabic{solution}:}\enspace\stepcounter{solution}}


\usepackage[letterpaper,top=2cm,bottom=2cm,left=3cm,right=3cm,marginparwidth=1.75cm]{geometry}
\usepackage{amsmath}
\usepackage{amssymb}
\usepackage{unicode-math}
\usepackage{graphicx}
\usepackage[colorlinks=true, allcolors=blue]{hyperref}
\begin{document}
\begin{figure}
\centering
\includegraphics[width=0.3\textwidth]{ifce.png}
\end{figure}

\title{Instituto de Educação de Ciências e Tecnologia do Ceará}
\author{José Francimi Vasconcelos de Queiroz Júnior}
\date{\today}



\maketitle

\section{Trabalho de Cálculo}
\subsection{Questões}
\begin{questions}

\question Sejam f e g funções tais que \(\lim_{x \to a}f(x) = L\) e \(\lim_{x \to a}g(x) = M\), com M $\neq$ 0. Prove que \(\lim_{x \to a} \frac{f(x)}{g(x)} = \frac{L}{M}\).

\answer{Como M $\neq$ 0 \Rightarrow Pelo teorema: \(\lim_{x \to a} \frac{1}{g(x)}=\frac{1}{M} \text{ e } \lim_{x \to a} f(x)=L \) $\therefore$ 
\newline \[\lim_ {x \to a} \frac{f(x)}{g(x)}=\lim_{x \to a} \left[f(x) \frac{1}{g(x)}\right] = \lim_{x \to a} \frac{f(x)}{g(x)} \lim_{x \to a} \frac{1}{g(x)}=L \frac{1}{M}=\frac{L}{M}\]}

\question Mostre que \(\lim_{x \to 3} x^2=9\).

\answer{Queremos provar que se \(0< \mid x-1 \mid <\) $\delta$ \( \Rightarrow \mid x^2-9 \mid < \) $\varepsilon$,
\newline mas \(\mid x^2-9 \mid = \mid (x-3)(x+3) \mid = \mid x-3 \mid \mid x+3 \mid\) 
\newline Suponha que $\delta$ \(\leq 1\) $\therefore$ se \(\mid x-3\mid <\) $\delta$ \Rightarrow \(\mid x-3\mid < 1\)
\newline \(-1 < x-3< 1 \Rightarrow 2 < x < 4 \Rightarrow 5 < x+3 <7 \Rightarrow \mid x+3 \mid <7\) 
\newline \newline $\therefore$ \(\mid x-3 \mid \mid x+3 \mid <\) $\delta$ 7 
\newline \newline $\delta$ \(\leq \frac{\text{$\varepsilon$}}{7}\) Tome $\delta$ \(= mín\{1, \frac{\text{$\varepsilon$}}{7}\}\) 
\newline \newline $\therefore$ Se \(0< \mid x-3 \mid <\) $\delta$ \Rightarrow \(\mid x-3 \mid < \frac{\text{$\varepsilon$}}{7}\) \Rightarrow \(\mid x-3 \mid 7 < \text{$\varepsilon$}\) \Rightarrow 
\newline \(\mid x-3 \mid \mid x+3 \mid < \mid x-3 \mid 7 < \text{$\varepsilon$}\) \Rightarrow \(\mid x-3 \mid \mid x+3 \mid <\) $\varepsilon$ \Rightarrow \(\mid x^2-9 \mid <\) $\varepsilon$ } 

\question Calcule:

\begin{parts}

\part \(\lim_{x \to -2} \frac{2x^3+9x^2+12x+4}{-x^3-2x^2+4x+8}\)

\answer{(a) Temos:\[\lim_{x \to -2} \frac{2x^3+9x^2+12x+4}{-x^3-2x^2+4x+8} \Rightarrow \lim_{x \to -2} \frac{(2x^2+5x+2)(x+2)}{(-x^2+4)(x+2)}\] 
 \[\Rightarrow \lim_{x \to -2} \frac{2x^2+5x+2}{-x^2+4} \Rightarrow \lim_{x \to -2} \frac{(2x+1)(x+2)}{(-x+2)(x+2)}\]
 \[\Rightarrow \lim_{x \to -2} \frac{2x+1}{-x+2} = \frac{-3}{4}\]
 \newline (b) Temos: \[\lim_{x \to 0} \frac{\sqrt{x+1}-\sqrt{1-x}}{x^2+x-2}\left( \frac{\sqrt{x+1}-\sqrt{1-x}}{\sqrt{x+1}-\sqrt{1-x}}  \right) =\]
 \[= \lim_{x \to 0} \left( \frac{x+1-(1-x)}{x(\sqrt{1+x}+\sqrt{1-x}}\right) = \lim_{x \to 0} \frac{2x}{x(\sqrt{1+x}+\sqrt{1-x})} = \lim_{x \to 0} \frac{2}{\sqrt{1+x}+\sqrt{1-x}} = 1\] } 

\part \(\lim_{x \to 0} \frac{\sqrt{x+1}-\sqrt{1-x}}{x^2+x-2}\)
%Resposta da b começa na linha 69
\end{parts}

\question Existe um número a tal que: \(\lim_{x \to -2} \frac{3x^2+ax+a+3}{x^2+x-2} \) exista? Caso exista, encontre a e o valor do limite.

\answer{Substituindo o x por -2 
\newline Temos: \(3(-2)^2+a(-2)+a+3=0\) \Rightarrow \(12-2a+a+3=0\) \Rightarrow \(-a+15=0\) \Rightarrow \(a=15\)
\newline O número existe e é 15
\newline Substituindo a por 15 temos: \[\lim_{x \to -2} \frac{3x^2+15x+18}{x^2+x-2} \Rightarrow \lim_{x \to -2} \frac{(3x+9)(x+2)}{(x-1)(x+2)}\] 
\newline \[\Rightarrow \lim_{x \to -2} \frac{3x+9}{x-1} = \frac{3(-2)+9}{-2-1} = \frac{3}{-3} = -1 \] }

\end{questions}

\theanswers

\clearpage


\end{document}
